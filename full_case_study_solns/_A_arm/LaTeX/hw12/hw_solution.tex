\begin{description}
\item[Step 1.]
The original state space equations are 
\begin{align*}
\dot{x} &= \begin{pmatrix} 0 &   1.0000 \\ 0 &  -0.667 \end{pmatrix}x + \begin{pmatrix} 0 \\ 66.667 \end{pmatrix} u \\
y &= \begin{pmatrix}1 & 0 \end{pmatrix}x,
\end{align*}
therefore the augmented system is
\begin{align*}
A_1 &= \begin{pmatrix} A & \mathbf{0} \\ -C & \mathbf{0} \end{pmatrix} = \begin{pmatrix} 0 &   1.0000 & 0 \\ 0 &  -0.667 & 0 \\ -1 & 0 & 0 \end{pmatrix} \\
B_1 &= \begin{pmatrix} B \\ \mathbf{0} \end{pmatrix} = \begin{pmatrix} 0 \\ 66.667 \\ 0 \end{pmatrix}
\end{align*}

\item[Step 2.] 
After the design in HW~\ref{ds:single_link_arm}.\ref{chap:state-feedback}, the closed loop poles were located at $p_{1,2} = -3.1191 \pm j 3.1200$.  We will add the integrator pole at $p_I=-5$.
The new controllability matrix is 
\begin{align*}
\mathcal{C}_{A_1,B_1} &= [B_1, A_1B_1, A_1^2B_1] \\ &= \begin{pmatrix} 
         0 &   66.6670 & -44.4669 \\
   66.6670 & -44.4669  & 29.6594 \\
         0 &        0  & -66.6670
\end{pmatrix}.
\end{align*}

The determinant is $det(\mathcal{C}_{A_1,B_1})=2.9630e+05\neq 0$, therefore the system is controllable.  

The open loop characteristic polynomial
\begin{align*}
\Delta_{ol}(s)&=\text{det}(sI-A_1) = \text{det}\begin{pmatrix} 
s &   -1.0000 & 0 \\ 
0 &  s+0.667 & 0 \\ 
1 & 0 & s
\end{pmatrix} \\ &= s^3 + 0.6174s^2,
\end{align*}
which implies that
\begin{align*}
\mathbf{a}_{A_1} &= \begin{pmatrix}0.6174, &        0,  &       0\end{pmatrix} \\
\mathcal{A}_{A_1} &= \begin{pmatrix} 
1 & 0.6174 & 0 \\ 0 & 1 & 0.6174 \\ 0 & 0 & 1
\end{pmatrix}.
\end{align*}

The desired closed loop polynomial
\begin{align*}
\Delta_{cl}^d(s) &= (s+3.8885-j3.8897)(s+3.8885+j3.8897)(s+5)\\
&=s^3+12.7770s^2+69.1350s+151.2500,
\end{align*}
which implies that
\[
\boldsymbol{\alpha} = \begin{pmatrix}12.7770, &  69.1350, &   151.2500\end{pmatrix}.
\]

The augmented gains are therefore given as
\begin{align*}
K_1 &= (\boldsymbol{\alpha}-\mathbf{a}_{A_1})\mathcal{A}_{A_1}^{-1}\mathcal{C}_{A_1,B_1}^{-1} \\
  &= \begin{pmatrix} 1.0370, &    0.1817, &   -2.2687 \end{pmatrix}
\end{align*}

\item[Step 3.]
The feedback gains are therefore given by
\begin{align*}
K &= K_1(1:2) = \begin{pmatrix} 1.0370, &    0.1817 \end{pmatrix} \\
k_I &= K_1(3) = -2.2687
\end{align*}

\end{description}

A Python class that implements state feedback control with an integrator for the single link robot arm is shown below.
\lstinputlisting[language=Python, caption=armController.py]{../control_book_public_solutions/_a_arm/python/hw12/armController.py}

Python code that computes the control gains is given below.
\lstinputlisting[language=Python, caption=armParamHW12.py]{../control_book_public_solutions/_a_arm/python/hw12/armParamHW12.py}

Complete simulation code for Matlab, Python, and Simulink can be downloaded at \controlbookurl{http://controlbook.byu.edu}.


%A Matlab class that implements state feedback control with an integrator for the single link robot arm is shown below.
%\lstinputlisting[language=Matlab, caption=armController.m]{../control_book_public_solutions/_A_arm/matlab/hw12/armController.m}
%
%Matlab code that computes the control gains is given below.
%\lstinputlisting[language=Matlab, caption=armParamHW12.m]{../control_book_public_solutions/_A_arm/matlab/hw12/armParamHW12.m}


