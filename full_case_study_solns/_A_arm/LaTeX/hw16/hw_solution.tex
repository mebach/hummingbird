{\bf (a)} The Bode plot of the plant $P(s)$, and the loop gain with PID control $P(s)C_{PID}(s)$ is shown in \fref{fig:hw_arm_bode_specs}.
\controlbookfigurefullpage{0.7}
	{6_design_studies/figures/hw_arm_bode_specs}
	{Bode plot for single link robot arm, plant only and under PID control.}
	{fig:hw_arm_bode_specs}
From \fref{fig:hw_arm_bode_specs} we see that below $\omega_r = 0.4$~rad/sec, the loop gain is above $B_r=39$~dB.  Therefore, from Equation~\eqref{eq:loop_tracking_Br} we have that
\[
\abs{e(t)} \leq \gamma_r \abs{r(t)},
\]
where $\gamma_r = 10^{-39/20} = 0.0112$, which implies that the tracking error will be 1.12\% of the magnitude of the input.


{\bf (b)} Suppose now that the desired reference input is $\theta^d(t) = 5t^2$.  Under PID control, the slope of the loop gain as $\omega\to 0$ is $-40$~dB/dec, which implies that the system is type~2 and will track a step and a ramp with zero steady state error.  For a parabola, there will be a finite error.  Since the loop gain under PID control is $B_2=20$~dB above the $1/s^2$ line, from Equation~\eqref{eq:loop:type_2_error_parabola} the steady state error satisfies
\[
\lim_{t\to\infty}\abs{e(t)} \leq \frac{A}{M_a}  = 5\cdot 10^{-20/20} = 0.5.
\]

{\bf (c)} If the input disturbance is below $\omega_{d_{in}} = 0.01$~rad/s, then the difference between the loop gain and the plant is $B_{d_{in}}=20$~dB at $\omega_{d_{in}} = 0.01$~rad/s, therefore, from Equation~\eqref{eq:loop_input_dist_rejection} we see that 
\[
\gamma_{d_{in}} = 10^{-20/20} = 0.1,
\]
implying that 10\% of the input disturbance will show up in the output.

{\bf (d)} For noise greater than $\omega_{no}=100$~rad/sec, we see from \fref{fig:hw_arm_bode_specs} that $B_n = 40$~dB.  Therefore, $\gamma_n = 10^{-40/20} = 0.01$ which implies that $1$\% of the noise will show up in the output signal.
