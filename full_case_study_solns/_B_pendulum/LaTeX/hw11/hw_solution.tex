
From HW~\ref{ds:pendulum}.\ref{chap:state_space_models}, the state space equations for the inverted pendulum are given by
\begin{align}
\dot{x} &= \begin{pmatrix}
       0 &        0 &   1.0000 &        0 \\
       0 &        0 &        0 &   1.0000 \\
       0 &  -1.7294 &   -0.0471 &        0 \\
       0 &  34.5882 &   0.1412 &        0 \\
\end{pmatrix} x + \begin{pmatrix}
     0 \\
     0 \\
0.9412 \\
-2.8235 \\
\end{pmatrix} u \notag \\
y &= \begin{pmatrix}
1 & 0 & 0 & 0 \\
0 & 1 & 0 & 0 
\end{pmatrix} x, \label{eq:hw_state_feedback_b1_1}
\end{align}
where Equation~\eqref{eq:hw_state_feedback_b1_1} represents the measured outputs.  The reference output is the position $z$, which implies that
\[
y_r = \begin{pmatrix} 1 & 0 & 0 & 0 \end{pmatrix} x.
\]
\begin{description}
\item[Step 1.] 
The controllability matrix is therefore
\begin{align*}
\mathcal{C}_{A,B} &= [B, AB, A^2B, A^3B] \\
	&= \begin{pmatrix}           
	0 &   0.9412 &  -0.0443  &  4.8851 \\
    0  & -2.8235 &  0.1329 & -97.6672 \\
    0.9412  & -0.0443 &  4.8851 &  -0.4597 \\
   -2.8235  & 0.1329 & -97.6672 & 5.2855 \end{pmatrix}.
\end{align*}
The determinant is $det(\mathcal{C}_{A,B})=-6104.1 \neq 0$, implying that the system is controllable.  
\item[Step 2.] The open loop characteristic polynomial is
\[
\Delta_{ol}(s)=\text{det}(sI-A) = s^4 + 0.0471 s^3 - 34.59 s^2 - 1.384 s
\]
which implies that
\begin{align*}
\mathbf{a}_A &= (0.0471, -34.59, -1.384, 0) \\
\mathcal{A}_A &= \begin{pmatrix} 
1 & 0.0471 & -34.59 & -1.384 \\ 0 & 1 & 0.0471 & -34.59 \\ 0 & 0 & 1 & 0.0471 \\ 0 & 0 & 0 & 1
\end{pmatrix}.
\end{align*}

\item[Step 3.] The desired closed loop polynomial is
\begin{align*}
\Delta_{cl}^d(s) &= (s^2+2\zeta_{\theta}\omega_{n_\theta} s + \omega_{n_\theta}^2)(s^2+2\zeta_{z}\omega_{n_z} s + \omega_{n_z}^2) \\
&=s^4 + 8.2955 s^3 + 34.4139 s^2 + 53.5334 s + 41.6455       
\end{align*}
which implies that
\[
\boldsymbol{\alpha} = ( 8.2955, 34.4139, 53.5334, 41.6455).
\]

\item[Step 4.]
The gains are therefore given as
\begin{align*}
%K &= (\boldsymbol{\alpha}-\mathbf{a}_A)\mathcal{A}_A^{-1}\mathcal{C}_{A,B}^{-1} \\
%  &= \begin{pmatrix} -0.2041  -17.1144   -0.5208   -2.4494 \end{pmatrix}
K &= (\boldsymbol{\alpha}-\mathbf{a}_A)\mathcal{A}_A^{-1}\mathcal{C}_{A,B}^{-1} 
= \begin{pmatrix} -1.5050,  -24.9399,   -1.9847,   -3.5829 \end{pmatrix}
\end{align*}
The feedforward reference gain $k_r=-1/C_r(A-BK)^{-1}B$ is computed using $C_r=(1,0,0,0)$, which gives
\begin{align*}
k_r &= \frac{-1}{C_r(A-BK)^{-1}B} 
  = -1.5050.
\end{align*}
\end{description}

A Python class that implements a state feedback controller for the inverted pendulum is shown below.
\lstinputlisting[language=Python, caption=pendulumController.py]{../control_book_public_solutions/_B_pendulum/python/hw11/pendulumController.py}

The gains are computed with the following Python script.
\lstinputlisting[language=Matlab, caption=pendulumParamHW11.py]{../control_book_public_solutions/_B_pendulum/python/hw11/pendulumParamHW11.py}

Complete simulation code for Matlab, Python, and Simulink can be downloaded at \controlbookurl{http://controlbook.byu.edu}.


