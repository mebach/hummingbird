
The Python code that can be used to generate similar plots is shown below.
\lstinputlisting[language=Python, caption=pendulumParamHW17.py]{../control_book_public_solutions/_B_pendulum/python/hw17/pendulumParamHW17.py}

The transfer functions for the inner and outer loop plants and controller are defined in Lines~12--20.  For this problem, we plot both the inner and outer loop frequency response on the same Bode plot, as implemented in Lines~24--25 and Lines~29--30. The results of this code are shown in \fref{fig:hw_pendulum_margins}.
\controlbookfigurefullpage{0.7}
	{6_design_studies/figures/hw_pendulum_margins}
	{The {\tt margin} and {\tt bode} plots for the open and closed loop systems of both the inner and outer loops of the inverted pendulum system.}
	{fig:hw_pendulum_margins}
As seen from \fref{fig:hw_pendulum_margins}, the bandwidth of the inner loop is approximately $17$~rad/sec, which is slightly larger than the cross over frequency of $11$~rad/sec.  
%
Similarly, \fref{fig:hw_pendulum_margins} indicates that the bandwidth of the outer loop is approximately $1.3$~rad/sec, which is slightly larger than the cross over frequency of $1.2$~rad/sec with a phase margin of $PM=55$~degrees.

For a second order system, making the step response 10 times faster implies that $t_r$ is divided by 10, or equivalently that $\omega_n$ is multiplied by 10.  A similar principle holds for high order systems.  To be 10 times faster, the closed loop bandwidth of the inner loop should be a decade higher on the Bode plot than the closed loop bandwidth of the outer loop.

%The bandwidth separation between the inner and outer loop is about one decade justifying the successive loop closure design approach.  
