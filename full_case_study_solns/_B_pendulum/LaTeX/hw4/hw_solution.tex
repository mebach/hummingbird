
The differential equations describing the inverted pendulum derived in HW~\ref{ds:pendulum}.3 are
\begin{align}
(m_1+m_2)\ddot{z} + m_1 \frac{\ell}{2} \ddot{\theta} \cos\theta  &=
	 m_1 \frac{\ell}{2} \dot{\theta}^2\sin\theta -b\dot{z} + F 
	 \label{eq:dm_pendulum_nonlinear_eq} \\
m_1 \frac{\ell}{2} \ddot{z} \cos\theta +  m_1 \frac{\ell^2}{3} \ddot{\theta} &=
	m_1 g \frac{\ell}{2} \sin\theta. \notag
\end{align}
The equilibria values $(z_e,\theta_e,F_e)$ are found by setting $\dot{z}=\ddot{z}=\dot{\theta}=\ddot{\theta}=0$ in Equation~\eqref{eq:dm_pendulum_nonlinear_eq} to obtain
\begin{align}
F_e &= 0 \label{eq:dm_pendulum_equilibria_1} \\ 
m_1g\frac{\ell}{2}\sin\theta_e &= 0.\label{eq:dm_pendulum_equilibria_2}
\end{align}
Therefore, any triple $(z_e, \theta_e, F_e)$ satisfying Equations~\eqref{eq:dm_pendulum_equilibria_1} and~\eqref{eq:dm_pendulum_equilibria_2} is an equilibria, or in other words $z_e$ can be any value, $F_e=0$ and $\theta_e = k\pi$, where $k$ is an integer.  

To linearize around $(z_e, \theta_e, F_e)$ where $k$ is an even integer, note that
\begin{align*}
\ddot{\theta}\cos\theta &\approx 
	\ddot{\theta}_e\cos\theta_e
		+ \frac{\partial}{\partial\theta}(\ddot{\theta}\cos\theta)\Big|_{(\ddot{\theta}_e,\theta_e)}(\theta-\theta_e)
%		\\ & \qquad
		+ \frac{\partial}{\partial\ddot{\theta}}(\ddot{\theta}\cos\theta)\Big|_{(\ddot{\theta}_e,\theta_e)}(\ddot{\theta}-\ddot{\theta}_e) \\
		&= \ddot{\theta}_e\cos\theta_e
		- (\ddot{\theta}_e\sin\theta_e)\tilde{\theta}
		+ (\cos\theta_e)\ddot{\tilde{\theta}} \\
		&= \ddot{\tilde{\theta}} \\
%
\dot{\theta}^2\sin\theta &\approx \dot{\theta}_e^2\sin\theta_e + \frac{\partial}{\partial\theta}(\dot{\theta}^2\sin\theta)\Big|_{(\theta_e,\dot{\theta}_e)}(\theta-\theta_e) 
%	\\ & \qquad
	+ \frac{\partial}{\partial\dot{\theta}}(\dot{\theta^2}\sin\theta)\Big|_{(\theta_e,\dot{\theta}_e)}(\dot{\theta}-\dot{\theta}_e) \\
&= \dot{\theta}_e^2\sin\theta_e + \dot{\theta}_e^2\cos\theta_e\tilde{\theta} + 2\dot{\theta}_e\sin\theta_e\dot{\tilde{\theta}} \\
&= 0, \\
%
\ddot{z}\cos\theta &\approx 
	\ddot{z}_e\cos\theta_e
		+ \frac{\partial}{\partial\theta}(\ddot{z}\cos\theta)\Big|_{(\ddot{z}_e,\theta_e)}(\theta-\theta_e)
%		\\ & \qquad
		+ \frac{\partial}{\partial\ddot{z}}(\ddot{z}\cos\theta)\Big|_{(\ddot{z}_e,\theta_e)}(\ddot{z}-\ddot{z}_e) \\
		&= \ddot{z}_e\cos\theta_e
		- (\ddot{z}_e\sin\theta_e)\tilde{\theta}
		+ (\cos\theta_e)\ddot{\tilde{z}} \\
		&= \ddot{\tilde{z}} \\
%
\sin\theta &\approx \sin\theta_e + \frac{\partial}{\partial \theta}
(\sin\theta)\Big|_{\theta_e} 
(\theta-\theta_e) \\
	&= \sin\theta_e + (\cos\theta_e) \tilde{\theta} \\
	&= \tilde{\theta},
\end{align*}
where we have defined
$\tilde{\theta}\defeq\theta-\theta_e$ and $\tilde{z}\defeq z-z_e$.  Also defining $\tilde{F}\defeq F-F_e$, and noting that
\begin{align*}
\theta &= \theta_e + \tilde{\theta} = \tilde{\theta} \\
\dot{\theta} &= \dot{\theta}_e + \dot{\tilde{\theta}} = \dot{\tilde{\theta}} \\
z &= z_e + \tilde{z} \\
\dot{z} &= \dot{z}_e + \dot{\tilde{z}} = \dot{\tilde{z}} \\
\ddot{z} &= \ddot{z}_e + \ddot{\tilde{z}} = \ddot{\tilde{z}} \\
F &= F_e + \tilde{F} = \tilde{F},
\end{align*}
we can write Equation~\eqref{eq:dm_pendulum_nonlinear_eq}
 in its linearized form as
\begin{align*}
(m_1+m_2)[\ddot{z}_e+\ddot{\tilde{z}}] + m_1 \frac{\ell}{2} [\ddot{\tilde{\theta}}]  &=
	 m_1 \frac{\ell}{2} [ 0 ] -b[\dot{z}_e+\dot{\tilde{z}}] + [F_e+\tilde{F}] 
	  \\
m_1 \frac{\ell}{2} [ \ddot{\tilde{z}}] +  m_1 \frac{\ell^2}{3} [\ddot{\theta}_e+\ddot{\tilde{\theta}}] &=
	m_1 g \frac{\ell}{2} [\tilde{\theta}],
\end{align*}
which simplifies to
\begin{equation}\label{eq:pendulum_linearized_eom}
\begin{pmatrix} (m_1+m_2) & m_1 \frac{\ell}{2} \\
m_1 \frac{\ell}{2} & m_1 \frac{\ell^2}{3} \end{pmatrix}\begin{pmatrix}
\ddot{\tilde{z}} \\ \ddot{\tilde{\theta}} \end{pmatrix}  =
\begin{pmatrix} -b\dot{\tilde{z}} + \tilde{F} 
	  \\
	m_1 g \frac{\ell}{2} \tilde{\theta} \end{pmatrix},
\end{equation}
which are the linearized equations of motion.
