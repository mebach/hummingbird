
From HW~\ref{ds:satellite}.\ref{chap:state_space_models}, the state space equations for the satellite are given by
\begin{align}
\dot{x} &= \begin{pmatrix}
         0  &       0 &   1.0000 &        0 \\
         0  &       0 &        0 &   1.0000 \\
   -0.0300  &  0.0300 &  -0.0100 &   0.0100 \\
    0.1500  & -0.1500 &   0.0500 &  -0.0500 
\end{pmatrix} x + \begin{pmatrix}
     0 \\
     0 \\
0.2 \\
     0 \\
\end{pmatrix} u \notag \\
y &= \begin{pmatrix}
1 & 0 & 0 & 0 \\
-1 & 1 & 0 & 0 
\end{pmatrix} x\label{eq:hw_state_feedback_c1}
\end{align}
where Equation~\eqref{eq:hw_state_feedback_c1} represents the measured outputs.  The reference output is the angle $\phi$, which implies that
\[
y_r = \begin{pmatrix} 0 & 1 & 0 & 0 \end{pmatrix} x.
\]
\begin{description}
\item[Step 1.] 
The controllability matrix is 
\begin{align*}
\mathcal{C}_{A,B} &= [B, AB, A^2B, A^3B] \\
 	&= \begin{pmatrix}           
        0  &  0.2000  & -0.0020  & -0.0059 \\
         0 &        0 &   0.0100 &   0.0294 \\
    0.2000 &  -0.0020 &  -0.0059 &   0.0007 \\
         0 &   0.0100 &   0.0294 &  -0.0036 \end{pmatrix}.
\end{align*}
The determinant is $det(\mathcal{C}_{A,B})=-36,000\neq 0$, therefore the system is controllable.  
\item[Step 2.] The open loop characteristic polynomial is
\[
\Delta_{ol}(s)=\text{det}(sI-A) = s^4 + 0.0600 s^3 + 0.18 s^2 
\]
which implies that
\begin{align*}
\mathbf{a}_A &= (0.06, 0.18, 0, 0) \\
\mathcal{A}_A &= \begin{pmatrix} 
1 & 0.06 & 0.18 & 0 \\ 0 & 1 & 0.06 & 0.18 \\ 0 & 0 & 1 & 0.06 \\ 0 & 0 & 0 & 1
\end{pmatrix}.
\end{align*}

\item[Step 3.] The desired closed loop polynomial is
\begin{align*}
\Delta_{cl}^d(s) &= (s^2+2\zeta_{\theta}\omega_{n_\theta} s + \omega_{n_\theta}^2)(s^2+2\zeta_{\phi}\omega_{n_\phi} s + \omega_{n_\phi}^2) \\
&=s^4 + 4.9275 s^3 + 12.1420 s^2 + 14.6702 s + 8.8637,              
\end{align*}
when $\omega_\theta = 1.9848$, $\zeta_\theta = 0.707$, $\omega_\phi=1.5$, $\zeta_\phi = 0.707$, 
which implies that
\[
\boldsymbol{\alpha} = (4.9275,   12.1420,   14.6702,    8.8637).
\]

\item[Step 4.]
The gains are therefore given by
\begin{align*}
K &= (\boldsymbol{\alpha}-\mathbf{a}_A)\mathcal{A}_A^{-1}\mathcal{C}_{A,B}^{-1} 
  = (40.2842,  255.1732,   24.3375,  366.1825)
\end{align*}
The feedforward reference gain $k_r=-1/C_r(A-BK)^{-1}B$ is computed using $C_r=(0,1,0,0)$, which gives
\begin{align*}
k_r &= \frac{-1}{C_r(A-BK)^{-1}B} 
    = 295.4573.
\end{align*}
\end{description}


A Python class that implements a PID controller for the satellite is shown below.
\lstinputlisting[language=Python, caption=satelliteController.py]{../control_book_public_solutions/_C_satellite/python/hw11/satelliteController.py}

The gains are computed with the following Python script. 
\lstinputlisting[language=Python, caption=satelliteParamHW11.py]{../control_book_public_solutions/_C_satellite/python/hw11/satelliteParamHW11.py}

Complete simulation code for Matlab, Python, and Simulink can be downloaded at \controlbookurl{http://controlbook.byu.edu}.


	
%The high-level Simulink diagram that implements a state feedback controller is simular to  Figure~\ref{fig:hw8_simulink_satellite}, and the parameter file that computes the gains is similar to that given in Section~\ref{ds:single_link_arm}.\ref{chap:state-feedback}.
%The Matlab/Simulink code for the controller is listed below.
%\lstinputlisting[language=Matlab, caption=satellite\_ctrl.m]{../control_book_public_solutions/_C_satellite/simulink/hw11/satellite_ctrl.m}
%Complete simulation code for Matlab, Python, and Simulink can be downloaded at \controlbookurl{http://controlbook.byu.edu}.

 