\begin{description}
\item[Step 1.]
The original state space equations are 
\begin{align*}
\dot{x} &= \begin{pmatrix}          0 &   1.0000 \\
   -0.6000 &  -0.1000 \end{pmatrix}x + \begin{pmatrix}  0 \\
    0.2000 \end{pmatrix} u \\
y &= \begin{pmatrix}1 & 0 \end{pmatrix}x,
\end{align*}
therefore the augmented system is
\begin{align*}
A_1 &= \begin{pmatrix} A & \mathbf{0} \\ -C & \mathbf{0} \end{pmatrix} = \begin{pmatrix}  0 &   1.0000 & 0 \\
   -0.6000 &  -0.1000 & 0  \\ -1 & 0 & 0 \end{pmatrix} \\
B_1 &= \begin{pmatrix} B \\ \mathbf{0} \end{pmatrix} = \begin{pmatrix} 0 \\ 0.2000 \\ 0 \end{pmatrix}
\end{align*}

\item[Step 2.] 
Following the design in HW~\ref{hw:mass}.\ref{chap:state-feedback}, we use the tuning parameters $t_r = 2.5$~seconds and $\zeta = 0.707$.  In addition, we will add an integrator pole at $p_I = -10$.
The new controllability matrix
\[
\mathcal{C}_{A_1,B_1} = [B_1, A_1B_1, A_1^2B_1] = \begin{pmatrix} 
         0 &   0.2000 &  -0.0200 \\
    0.2000 &  -0.0200 &  -0.1180 \\
         0 &        0 &  -0.2000
\end{pmatrix}.
\]
The determinant is $det(\mathcal{C}_{A_1,B_1})=0.0080\neq 0$, therefore the system is controllable.  

The open loop characteristic polynomial
\begin{align*}
\Delta_{ol}(s)&=\text{det}(sI-A_1) = \text{det}\begin{pmatrix} 
        s  &  -1.0000    &     0 \\
   0.6000  & s+0.1000    &     0 \\
   1.0000  &       0     &    s \\
\end{pmatrix} \\ &= s^3 + 0.1000s^2 + 0.6000 s,
\end{align*}
which implies that
\begin{align*}
\mathbf{a}_{A_1} &= \begin{pmatrix}0.1000, &        0.6,  &       0\end{pmatrix} \\
\mathcal{A}_{A_1} &= \begin{pmatrix} 
1 & 0.1 & 0.6 \\ 0 & 1 & 0.1 \\ 0 & 0 & 1
\end{pmatrix}.
\end{align*}

The desired closed loop polynomial
\begin{align*}
\Delta_{cl}^d(s) &= (s^2+2\zeta\omega_ns+\omega_n^2)(s+10)\\
&=s^3+11.2443s^2+13.2176s+7.7440,
\end{align*} 
which implies that
\[
\boldsymbol{\alpha} = \begin{pmatrix}11.2443, &   13.2176, &    7.7440\end{pmatrix}.
\]

The augmented gains are therefore given as
\begin{align*}
K_1 &= (\boldsymbol{\alpha}-\mathbf{a}_{A_1})\mathcal{A}_{A_1}^{-1}\mathcal{C}_{A_1,B_1}^{-1} \\
  &= \begin{pmatrix} 63.0880, &   55.7216, &  -38.7200 \end{pmatrix}
\end{align*}

\item[Step 3.]
The feedback gains are therefore given by
\begin{align*}
K &= K_1(1:2) = \begin{pmatrix} 63.0880, &   55.7216 \end{pmatrix} \\
k_I &= K_1(3) = -38.7200
\end{align*}

\end{description}


% ckp 12/28/20: it looks like this was started to switch solutions between simulink and Matlab.  I have been swapping in all the Python code as I update the solutions.  This would probably be better since it is more versitile.  But, this is the first instance I have seen this, so for now will just go with Python.  On another iteration of cleaning up the solutions will provide a toggle between the three code types we have.
% 
%Alternatively, we could have used the following Matlab script
%\iftoggle{soln}{%
%  \lstinputlisting{simulink_d12/param.m}
%}{%
%  \lstinputlisting{./6_design_studies/_D_mass/simulink/hw12/massParamHW12.m}
%}
%
%Matlab code that implements the associated controller listed below.
%\iftoggle{soln}{%
%  \lstinputlisting{simulink_d12/mass_ctrl.m}
%}{%
%  \lstinputlisting{./6_design_studies/_D_mass/simulink/hw12/mass_ctrl.m}
%}


Alternatively, we could have used the following Python script
\ifsolutionmanual
\lstinputlisting{./6_design_studies/_D_mass/python/hw12/massParamHW12.py}
\else
\lstinputlisting{../../python/hw12/massParamHW12.py}
\fi

Python code that implements the associated controller listed below.
\ifsolutionmanual
\lstinputlisting{./6_design_studies/_D_mass/python/hw12/massController.py}
\else
\lstinputlisting{../../python/hw12/massController.py}
\fi


The complete simulation files are contained on the wiki associated with this book.

