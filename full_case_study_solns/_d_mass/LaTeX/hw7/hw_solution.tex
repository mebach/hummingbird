
The open loop transfer function from homework D.6 is
\[
Z(s) = \left(\frac{\frac{1}{m}}{s^2+\frac{b}{m}s+\frac{k}{m}}\right)F(s).  
\]
Using the parameters from Section~\ref{hw:mass} gives
\[
Z(s) = \left(\frac{0.2}{s^2+0.1s+0.6}\right)F(s).  
\]
The open loop poles are therefore the roots of the open loop polynomial
\[
\Delta_{ol}(s) = s^2+0.1s+0.6,
\]
which are given by
\[
p_{ol} = -0.05\pm j 0.773.
\]
Using PD control, the closed loop system is therefore shown in Figure~\ref{fig:hw_mass_block_diagram_PD}.
\begin{figure}[H]
   \centering
   \includegraphics[width=0.7\textwidth]{6_design_studies/figures/hw_mass_block_diagram_PD.pdf}
   \caption{PD control of the mass spring damper.}
   \label{fig:hw_mass_block_diagram_PD}
\end{figure}
The transfer function of the closed loop system is given by
\begin{align*}
& Z(s) = \left(\frac{0.2}{s^2+0.1s+0.6}\right)\left(k_P (Z_r(s)-Z(s)) - k_D s Z(s) \right)  \\
\implies & (s^2+0.1s+0.6)Z(s) = \left(0.2 k_P (Z_r(s)-Z(s)) - 0.2 k_D s Z(s) \right) \\
\implies & \left[(s^2+0.1s+0.6)+\left(0.2k_D s +0.2k_P\right)\right]Z(s) = 0.2k_P Z_r(s)  \\
\implies & \left[s^2 + (0.1 + 0.2k_D)s + (0.6+0.2k_P)\right]Z(s) = 0.2k_P Z_r(s)  \\
\implies & Z(s) = \frac{0.2k_P}{s^2 + (0.1 + 0.2k_D)s + (0.6+0.2k_P)}Z_r(s). 
\end{align*}
Therefore, the closed loop poles are given by the roots of the closed loop characteristic polynomial
\[
\Delta_{cl}(s) = s^2 + (0.1 + 0.2k_D)s + (0.6+0.2k_P)
\]
which are given by
\[
p_{cl} = -\frac{(0.1 + 0.2k_D)}{2} \pm \sqrt{ \left(\frac{(0.1 + 0.2k_D)}{2}\right)^2 - \left(0.6+0.2k_P\right)}
\]
If the desired closed loop poles are at $-1$ and $-1.5$, then the desired closed loop characteristic polynomial is
\begin{align*}
\Delta_{cl}^d &= (s+1.5)(s+1) \\
 &= s^2 + 2.5s + 1.5.
\end{align*}
Equating the actual closed loop characteristic polynomial $\Delta_{cl}$ with the desired characteristic polynomial $\Delta_{cl}^d$ gives
\[
s^2 + (0.1 + 0.2k_D)s + (0.6+0.2k_P) = s^2 + 2.5s + 1.5,
\]
or by equating each term we get 
\begin{align*}
0.1 + 0.2k_D &=2.5 \\
0.6 + 0.2k_P &= 1.5.
\end{align*}
Solving for $k_P$ and $k_D$ gives
\begin{align*}
k_P &= 4.5 \\
k_D &= 12
\end{align*}

The Simulink implementation is contained on the wiki associated with this book.