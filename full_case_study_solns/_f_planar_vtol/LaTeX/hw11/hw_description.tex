The objective of this problem is to implement a state feedback controller for the VTOL system.  
Start with the simulation files developed in Homework~\ref{hw:vtol}.\ref{chap:PID-digital-implementation}.
\begin{description}
\item[(a)] Using the values for $\omega_{n_h}$, $\zeta_h$, $\omega_{n_z}$, and $\zeta_z$ from Homework~\ref{hw:vtol}.\ref{chap:PID-design-specs}, choose the closed-loop pole locations to ensure that the longitudinal poles have damping ratios greater than $\zeta_h$ and natural frequencies greater than $\omega_{n_h}$. Similarly choose the closed-loop poles for the lateral dynamics to ensure that their damping ratios are greater than $\zeta_z$ and natural frequencies greater than $\omega_{n_z}$
\item[(b)] Add the state space matrices $A$, $B$, $C$, $D$ derived in Homework~\ref{hw:vtol}.\ref{chap:state_space_models} to your param file.
\item[(c)] Verify that the state space system is controllable by checking that $\text{rank}(\mathcal{C}_{A,B})=n$.
\item[(d)] Find the feedback gain $K$ such that the eigenvalues of $(A-BK)$ are equal to desired closed loop poles.  Find the reference gains $k_{r_h}$ and $k_{r_z}$ so that the DC-gain from $h_r$ to $h$ is equal to one, and the DC-gain from $z_r$ to $z$ is equal to one.  
\item[(e)] Implement the state feedback scheme and tune the closed-loop poles to get acceptable response. What changes would you make to your closed-loop pole locations to decrease the rise time of the system? How would you change them to lower the overshoot in response to a step command?
\end{description}
