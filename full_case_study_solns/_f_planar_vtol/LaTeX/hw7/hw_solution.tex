
The open-loop transfer function from homework F.5 is
\[
\tilde{H}(s) = \left(\frac{\frac{1}{m_c+2m_r}}{s^2}\right)\tilde{F}(s).  
\]
Using the parameters from Section~\ref{hw:vtol} gives
\[
\tilde{H}(s) = \left(\frac{0.667}{s^2}\right)\tilde{F}(s).  
\]
The open-loop poles are therefore the roots of the open loop polynomial
\[
\Delta_{ol}(s) = s^2,
\]
which are given by
\[
p_{ol} = 0, 0.
\]
Using PD control, the closed-loop system is therefore shown in Figure~\ref{fig:hw_vtol_PD_altitude}.
\begin{figure}[H]
   \centering
   \includegraphics[width=0.7\textwidth]{6_design_studies/figures/hw_vtol_PD_altitude.pdf}
   \caption{PD control of the altitude control for the planar VTOL system.}
   \label{fig:hw_vtol_PD_altitude}
\end{figure}
The transfer function of the closed-loop system is given by
\begin{align*}
& \tilde{H}(s) = \left(\frac{0.667}{s^2}\right)\left(k_P (\tilde{H}_r(s)-\tilde{H}(s)) - k_D s \tilde{H}(s) \right)  \\
\implies & (s^2)\tilde{H}(s) = \left(0.667 k_P (\tilde{H}_r(s)-\tilde{H}(s)) - 0.667 k_D s \tilde{H}(s) \right) \\
\implies & (s^2 + 0.667k_Ds + 0.667k_P)\tilde{H}(s) = 0.667k_P \tilde{H}_r(s)  \\
\implies & \tilde{H}(s) = \frac{0.667k_P}{s^2 + 0.667k_Ds + 0.667k_P}\tilde{H}_r(s). 
\end{align*}
Therefore, the closed-loop poles are given by the roots of the closed-loop characteristic polynomial
\[
\Delta_{cl}(s) = s^2 + 0.667k_Ds + 0.667k_P.
\]
If the desired closed-loop poles are at $-0.3$ and $-0.2$, then the desired closed loop characteristic polynomial is
\begin{align*}
\Delta_{cl}^d &= (s+0.3)(s+0.2) \\
 &= s^2 + 0.5s + 0.06.
\end{align*}
Equating the actual closed-loop characteristic polynomial $\Delta_{cl}$ with the desired characteristic polynomial $\Delta_{cl}^d$ gives
\[
s^2 + 0.667k_Ds + 0.667k_P = s^2 + 0.5s + 0.06,
\]
or by equating each term we get 
\begin{align*}
k_P &= 0.09 \\
k_D &= 0.75.
\end{align*}

The Simulink implementation is contained on the wiki associated with this book.